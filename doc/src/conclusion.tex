\section{Conclusiones y Recomendaciones}

Durante el desarrollo del proyecto se logro implementar una 
aplicación que permite el acceso y el control de un sistema 
de restaurante, en este mismo le pueden manejar los pedidos 
de los clientes, el manejo del inventario y de las comidas
que se ofrecen en el restaurante. Con el fin de cumplir con 
los requerimientos del proyecto y obtener el mejor resultado 
posible con el tiempo que se tenía para realizarlo, se decidió 
optar por alternativas que permitieran el desarrollo de la 
aplicación de una manera más rápida y eficiente.

Por último, se puede concluir que el desarrollo de este proyecto permitió
la aplicación de los conocimientos adquiridos durante el curso de Lenguajes 
de Programación, en el cual se aprendió sobre el uso de diferentes lenguajes 
de programación, así como también el uso de diferentes herramientas que 
permiten el desarrollo de aplicaciones de una manera más eficiente. En este caso 
el uso de Prolog para generar una gran cantidad de combinaciones de comidas, 
el uso de Python para el uso de patrones de diseño y la interconexión de 
los diferentes componentes de la aplicación, todo esto y más permitió 
el desarrollo de una aplicación que cumple con los requerimientos del 
proyecto y que permite el manejo de un sistema de restaurante de una 
manera eficiente y sencilla. De esta manera se puede concluir que el 
desarrollo de este proyecto fue exitoso y que se logró cumplir con 
los objetivos planteados al inicio del mismo.
